%!TEX TS-program = xelatex
%!TEX encoding = UTF-8 Unicode

\documentclass[paper=a4, fleqn]{article}
\usepackage[utf8]{inputenc}
\usepackage{xspace}
\usepackage{relsize}
\usepackage{amsmath}
\usepackage{url}

\newcommand{\clrs}{\textsmaller{CLRS}\xspace}

\title{Ugeopgave 1}
\author{Mathias Bramming \and Philip Meulengracht \and Rasmus Borgsmidt}
\date{}

\begin{document}
\maketitle
\section*{Task 1}
I denne opgave bedes vi benytte mestermetoden ({\em master method}) til at
angive øvre og nedre grænser for $p(n)$ i hvert af de følgende tilfælde.

\subsection*{$p(n)=8p(n/2)+n^2$}

Vi ser, at $T(n)=aT(n/b)+f(n)=8p(n/2)+n^2$, hvor $a=8, b=2, f(n)=n^2$. Da vi
har $f(n)=n^2=O(n^{\log_b a-\epsilon})=O(n^{\log_2(8)-\epsilon})=O(n^2)$, hvor
$\epsilon=1$, er $T(n)$ ifølge mestermetoden asymptotisk begrænset med:
\[
T(n)=\Theta(n^{\log_b a})=\Theta(n^3).
\]

\subsection*{$p(n)=8p(n/4)+n^3$}

Vi ser, at $T(n)=aT(n/b)+f(n)=8p(n/4)+n^3$, hvor $a=8, b=4, f(n)=n^3$. Da vi har
$f(n)=n^3=\Omega(n^{\log_b
  a+\epsilon})=\Omega(n^{\log_4(8)+\epsilon})=\Omega(n^3)$, hvor $\epsilon=3/2$,
og da $af(n/b)=n^3/8\leq cf(n)=n^3/2$, hvor $c=1/2, n\geq 0$, er $T(n)$ ifølge
mestermetoden asymptotisk begrænset med:
\[
T(n)=\Theta(f(n))=\Theta(n^3).
\]

\subsection*{$p(n)=10p(n/9)+n\log_2 n$}

Ikke udført.

\section*{Task 2}

I denne opgave bedes vi benytte indsætningsmetoden ({\em substitution method})
til at angive øvre og nedre grænser for $p(n)$ i hvert af de følgende tilfælde.

\subsection*{$p(n)=p(n/2)+p(n/3)+n$}

Rekursionsligningen for $p(n)$ ligner meget (4.19) fra \clrs, så vi starter
med følgende gæt:
\[
p(n)=\Theta(n\lg n)
\]
Vi ved fra opgaveteksten, at $p(1)=1$ og $p(2)=2$, så vi ser på tilfældet $n\geq
3$.

For den øvre grænse $p(n)=O(n\lg n)$ skal vi vise, at $p(n)\leq cn\lg n$ for en
konstant $c>0$ efter eget valg. Dette gøres ved induktion over $n$ , hvor alle
$n$ rundes ned til nærmeste heltal.

Vi vælger basistilfælde $p(3)=p(1)+p(1)+3 = 5$ og $p(4)=p(2)+p(1)+4=7$. Med valg
af f.eks.\ $c\geq 2$ opfyldes $p(n)\leq cn\lg n$ for $n\in\{3,4\}$.

Vi antager herefter, at $p(m)\leq cm\lg m$ for alle $3\leq m<n$; herunder
specielt $p(\lfloor n/2\rfloor)\leq c\lfloor n/2\rfloor\lg \lfloor n/2\rfloor$
og $p(\lfloor n/3\rfloor)\leq c\lfloor n/3\rfloor\lg \lfloor n/3\rfloor$.

Induktionsskridtet vises således:
\begin{align*}
p(n) &= p(\lfloor n/2\rfloor)+p(\lfloor n/3\rfloor)+n \\
     &\leq c\lfloor n/2\rfloor\lg \lfloor n/2\rfloor + c\lfloor n/3\rfloor\lg
     \lfloor n/3\rfloor + n \\
     &\leq c(n/2)\lg(n/2) + c(n/3)\lg(n/3) + n \\
     &= c(n/2)\lg n - c(n/2)\lg 2 + c(n/3)\lg n - c(n/3)\lg 3 + n \\
     &= (5/6)cn\lg n - cn(1/2 + \lg(3)/3) + n \\
     &\leq (5/6)cn\lg n - cn + n \\
     &\leq cn\lg n
\end{align*}

For den nedre grænse $p(n)=\Omega(n\lg n)$ skal vi vise, at $0\leq cn\lg n\leq p(n)$
for en konstant $c>0$ efter eget valg. Dette gøres ved induktion over $n$, hvor
alle $n$ rundes op til nærmeste heltal.

Vi vælger igen samme basistilfælde. Med valg af f.eks.\ $c=1$ opfyldes
$0\leq cn\lg n\leq p(n)$ for $n\in\{3,4\}$.

Vi antager herefter, at $0\leq cm\lg m\leq p(m)$ for alle $3\leq m<n$; herunder
specielt $0\leq c\lceil n/2\rceil\lg \lceil n/2\rceil\leq p(\lceil
n/2\rceil)$ og $0\leq c\lceil n/3\rceil\lg \lceil n/3\rceil\leq p(\lceil
n/3\rceil)$.

Induktionsskridtet vises således:
\begin{align*}
p(n) &= p(\lceil n/2\rceil)+p(\lceil n/3\rceil)+n \\
     &\geq c\lceil n/2\rceil\lg \lceil n/2\rceil + c\lceil n/3\rceil\lg
     \lceil n/3\rceil + n \\
     &\geq c(n/2)\lg(n/2) + c(n/3)\lg(n/3) + n \\
     &= c(n/2)\lg n - c(n/2)\lg 2 + c(n/3)\lg n - c(n/3)\lg 3 + n \\
     &= (5/6)cn\lg n - cn(1/2 + \lg(3)/3) + n \\
     &\geq (5/6)cn\lg n,\quad c < 1/2 + \lg(3)/3 = 1.0283 \\
     &= c'n\lg n,\quad\qquad c' < (5/6)\cdot 1.0283 = 0.8569 \\
\end{align*}

Så ved at vælge f.eks.\ $c=5/6$ opnås det ønskede resultat.

\subsection*{$p(n)=\sqrt{n}+p(\sqrt{n})+\sqrt{n}$}

Ikke udført.

\subsection*{$p(n)=2p(n-2)+n$}

Ikke udført.

\section*{Task 3}

Ikke udført.

\section*{Task 4}

Ikke udført.

\section*{Task 5}

Fordi heap-sort arbejder {\em in-place}. Mergesort har f.eks. N ekstra plads i
worst case (se evt skema \url{http://en.wikipedia.org/wiki/Sorting_algorithm})

\section*{Task 6}

In the case that the array is fully sorted, only $n-1$ comparisons have to be
made, which gives insertion sort a best-case runtime of $O(n)$. Both the average
and worst case of insertion sort are $O(n^2)$, but since introsort only switches
to insertion sort when a certain level of deepness is reached, those cases can
be avoided. Insertion sort works with low overhead, and is exceptional at
sorting very small lists (typically with less than 20 elements (evt reference?))
or nearly sorted lists.

\paragraph{}

Vi fik først sent etableret en afleveringsgruppe og har derfor ikke fået lavet
så meget af den første opgave. Vi håber derfor på at få mulighed for at lave en
genaflevering.

\end{document}
